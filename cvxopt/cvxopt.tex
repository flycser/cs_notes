\newif\ifColor
% Add \Colortrue to settings.tex to use color
% Otherwise, will show a primarily grayscale version of the document for printing

\documentclass[11pt,oneside,a4paper]{book}
% \usepackage{natbib}
\usepackage[numbers]{natbib}
\usepackage{breakcites} % Do Not let citations break out of the text frame
\usepackage{microtype} % Get rid of some frame busts automatically
% Note: if microtype causes error on ubuntu, run
% sudo apt-get install cm-super
\title{Notes on Convex Optimization}
\author{Fei Jie}
\date{}
\setcounter{tocdepth}{1}

\pdfobjcompresslevel=0

\usepackage{zref-abspage}

\setcounter{secnumdepth}{3} % Number subsubsections, because we reference them,
% so the reader needs numbers to find the correct place.


\usepackage[vcentering,dvips]{geometry}
\geometry{papersize={7in,9in},bottom=3pc,top=5pc,left=5pc,right=5pc,bmargin=4.5pc,footskip=18pt,headsep=25pt}


%%% Packages %%%
\usepackage{epsfig}
\usepackage{subfigure}
\usepackage[utf8]{inputenc}

% Needed for some foreign characters
\usepackage[T1]{fontenc}

\usepackage{amsmath}
\usepackage{subfigure}
\usepackage{amsfonts}
\usepackage{amsthm}
\usepackage{mathtools}
\usepackage{multirow}
\usepackage{colortbl}
\usepackage{booktabs}
% This allows us to cite chapters by name, which was useful for making the
% acknowledgements page
\usepackage{nameref}
% Make sure there is a space between the subsection number and subsection title
% in the table of contents.
% If we do not do this we end up with 2 digit subsection numbers colliding with
% the title.
% See https://tex.stackexchange.com/questions/7853/toc-text-numbers-alignment/7856#7856?newreg=d2632892dd0345f388619f12fa794b11
\usepackage[tocindentauto]{tocstyle}
\usetocstyle{standard}

\usepackage{bm}

\usepackage{float}
\newcommand{\boldindex}[1]{\textbf{\hyperpage{#1}}}
\usepackage{makeidx}\makeindex
% Make bibliography and index appear in table of contents
\usepackage[nottoc]{tocbibind}
% Using the caption package allows us to support captions that contain "itemize" environments.
% The font=small option makes the text of captions smaller than the main text.
\usepackage[font=small]{caption}

% Used to change header sizes
\usepackage{fancyhdr}

\usepackage[chapter]{algorithm}
\usepackage{algorithmic}
% Include chapter number in algorithm number
\renewcommand{\thealgorithm}{\arabic{chapter}.\arabic{algorithm}}


\theoremstyle{definition}
\newtheorem{example}{Example}[section]

% Define the P table cell environment
% It is the same as p, but centers the text horizontally
\usepackage{array}
\newcolumntype{P}[1]{>{\centering\arraybackslash}p{#1}}

% Rebuild the book document class's headers from scratch, but with different font size
% (this is for MIT Press style)
% Source: http://texblog.org/2012/10/09/changing-the-font-size-in-fancyhdr/
\newcommand{\changefont}{% 1st arg to fontsize is font size. 2nd arg is the baseline skip. both in points.
    \fontsize{9}{11}\selectfont
}
\fancyhf{}
%\fancyhead[LE,RO]{\changefont \slshape \rightmark} %section
\fancyhead[RE,LO]{\changefont \slshape \leftmark} %chapter
\fancyfoot[C]{\changefont \thepage} %footer
\pagestyle{fancy}
\input{math_commands.tex}

\usepackage[pdffitwindow=false,
pdfview=FitH,
pdfstartview=FitH,
pagebackref=true,
breaklinks=true,
\ifColor
colorlinks,
\fi
bookmarks=false,
plainpages=false]{hyperref}

% Make \[ \] math have equation numbers
\DeclareRobustCommand{\[}{\begin{equation}}
\DeclareRobustCommand{\]}{\end{equation}}

% Allow align environments to cross page boundaries.
% If we do not do this, we get weird gaps of several inches of white space
% before or after some long align environments.
\allowdisplaybreaks

\begin{document}

\setlength{\parskip}{0.25 \baselineskip}
\newlength{\figwidth}
\setlength{\figwidth}{26pc}
% Spacing between notation sections
\newlength{\notationgap}
\setlength{\notationgap}{1pc}

\typeout{START_CHAPTER "TOC" \theabspage}
\frontmatter


\maketitle
\tableofcontents
\typeout{END_CHAPTER "TOC" \theabspage}

\chapter{Preface}

This document is notes on the textbook Convex Optimization \cite{boyd2004convex}.

\mainmatter

\chapter{Introduction}

\section{Mathematical optimization}

A mathematical optimization problem has the form
\begin{equation}\label{prob:opt}
	\begin{split}
		\text{minimize}\quad   & f_0(x) \\
		\text{subject to}\quad & f_i(x)\leq b_i,\quad i=1,\dots ,m \\
	\end{split}
\end{equation}
Here the vector $x=(x_1,\dots,x_n)$ is the optimization variable of the problem, the function $ f_0:\R^n\to\R $ is the objective function, the functions $ f_i:\R^n\to\R, i=1,\dots,m $ are the limits, or bounds, for the constraints. A vector $ x^* $ is called optimal, or a solution of the problem (\ref{prob:opt}), if it has the smallest objective value among all vectors that satisfy the constraints: for any $ z $ with $ f_1(z)\leq b_1,\dots,f_m(z)\leq b_m $, we have $ f_0(z)\geq f_0(x^*) $.


\section{Least squares and linear programming}

\subsection{Least-squares problems}

A least-squares problem is an optimization problem with no constraints and an objective which is a sum of squares of terms of the form $ a_i^T x-b_i $:
\begin{equation}
\text{minimize}\quad f_0(x)=\|Ax-b\|_2^2=\sum_{i=2}^k (a_i^Tx-b_i)^2.
\end{equation}
Here $ A\in\R^{k\times n} $ (with $ k\geq n $), $ a_i^T $ are the rows of $ A $, and the vectors $ x\in\R^n $ is the optimization variable.

\subsection{Linear programming}

In linear programming problem, the objective function and all constraint functions are linear:
\begin{equation}
\begin{split}
\text{minimize}\quad&c^Tx \\
\text{subject to}\quad& c_i^Tx\leq b_i,\quad i=1,\dots, m.
\end{split}
\end{equation}
Here the vectors $ c,a_1,\dots,a_m\in\R^n $ and scalars $ b_1,\dots,b_m\in\R $ are problem parameters that specify the obejctive and constraint functions.

\section{Convex optimization}

A convex optimization problem is one of the form
\begin{equation}
\begin{split}
\text{minimize}\quad &f_0(x) \\
\text{subject to}\quad& f_i(x)\leq b_i,\quad i=1,\dots,m,
\end{split}
\end{equation}
where the functions $ f_0,\dots,f_m:\R^n\to\R $ are convex, i.e., satisfy
\begin{equation}
f_i(\alpha x+\beta y)\leq \alpha f(x)+\beta f(y)
\end{equation}
for all $ x,y\in\R^n $ and all $ \alpha,\beta\in\R $ with $ \alpha+\beta=1,\alpha\geq 0, \beta\geq 0 $.n

\section{Nonlinear optimization}

\section{Outline}

\section{Notation}

\part{Theory}

\chapter{Convex sets}  

\section{Affine and convex sets}

\subsection{Lines and line segments}

Suppose $ x_1\neq x_2 $ are two different points in $ \R^n $. Points of the form
\begin{equation*}
y=\theta x_1+(1-\theta) x_2,
\end{equation*}
where $ \theta\in\R $, form the line passing through $ x_1 $ and $ x_2 $.

\subsection{Affine sets}

A set $ C $ is affine if the line through any two distinct points in $ C $ lies in $ C $, i.e., for any $ x_1,x_2\in C $ and $ \theta\in\R $, we have $ \theta x_1+(1-\theta)x_2\in C $.

\subsection{Affine dimension and relative interior}

\subsection{Convex sets}

A set $ C $ is convex if the line segment between any two points in $ C $ lies in $ C $, i.e., if for any $ x_1,x_2\in C $ and any $ \theta $ with $ 0\leq\theta\leq1 $, we have
\begin{equation*}
\theta x_1+(1-\theta)x_2\in C.
\end{equation*}

\subsection{Cones}

A set $ C $ is called a cone, or nonnegative homogeneous, if for every $ x\in C $ and $ \theta\geq 0 $, we have $ \theta x\in C $. A set $ C $ is a convex cone if it is convex and a cone, which means that for any $ x_1,x_2\in C $ and $ \theta_1,\theta_2\geq 0 $, we have
\begin{equation*}
\theta_1 x_1+\theta_2 x_2\in C.
\end{equation*}


\section{Some important examples}

\subsection{Hyperplanes and halfspaces}

\subsection{Euclidean balls and ellipsoids}

\subsection{Norm balls and norm cones}

\subsection{Polyhedra}

\subsection{The positive semidefinite cones}

\section{Operations that preserve convexity}

\subsection{Intersection}

\subsection{Affine functions}

\subsection{Linear-fractional and perspective functions}

\section{Generalized inequalities}

\subsection{Proper cones and generalized inequalities}

A cone $ K\subseteq \R^n $ is called a proper cone if it satisfies the following:
\begin{itemize}
\item $ K $ is convex.
\item $ K $ is closed.
\item $ K $ is solid, which means it has nonempty interior.
\item $ K $ is pointed, which means that it contains no line (or equivalently, $ x\in K, -x\in K\Rightarrow x=0 $).
\end{itemize}
A proper cone $ K $ can be used to define a generalized inequality, which is a partial order on $ \R^n $ that has many of the properties of the standard ordering on $ \R $.

\subsection{Minimum and minimal elements}

The most obvious difference between ordinary inequalities and generalized inequalities is that $ \leq $ for $ \R $ is a linear ordering: any two points are comparable, meaning either $ x\leq y $ or $ y\leq x $. This property does not hold for other generalized inequalities.

$ x\in S $ is the minimum element of $ S $ (with respect to the generalized inequality $ \preceq_K $) if for every $ y\in S $ we have $ x\preceq_K y $. If a set has minimum (maximum) element, then it is unique. $ x\in S $ is a minimal element of $ S $ (with respect to the generalized inequality $ \preceq_K $) if $ y\in S $, $ y\preceq_K x $ only if $ y=x $.

\section{Separating and supporting hyperplanes}

\section{Dual cones and generalized inequalities}

\chapter{Convex functions}

\section{Basic properties and examples}

\section{Operations that preserve convexity}

\section{The conjugate function}

\section{Quasiconvex functions}

\section{Log-concave and log-convex functions}

\section{Convexity with respect to generalized inequalities}

\chapter{Convex optimization problems}

\section{Optimization problems}

\section{Convex optimization}

\section{Linear optimization problems}

\section{Quadratic optimization problems}

\section{Geometric programming}

\section{Generalized inequality constraints}

\section{Vector optimization}

\chapter{Duality}

\section{The Lagrange dual function}

\section{The Lagrange dual problem}

\section{Geometric interpretation}

\section{Saddle-point interpretation}

\section{Optimality conditions}

\section{Perturbation and sensitivity analysis}

\section{Examples}

\section{Theorems of alternatives}

\section{Generalized inequalities}

\part{Applications}

\chapter{Approximation and fitting}

\chapter{Statistical estimation}

\chapter{Geometric problems}

\part{Algorithms}

\chapter{Unconstrained minimization}

\chapter{Equality constrained minimization}

\chapter{Interior-point methods}

\appendix

\part*{Appendices}
\addcontentsline{toc}{part}{Appendices}

\chapter{Mathematical background}

\chapter{Problems involving two quadratic functions}

\chapter{Numerical linear algebra background}

\small{
\typeout{START_CHAPTER "bib" \theabspage}
\bibliography{../references}
% \bibliographystyle{natbib}
\bibliographystyle{acm}
\clearpage
\typeout{END_CHAPTER "bib" \theabspage}
}
\typeout{START_CHAPTER "index-" \theabspage}
\printindex
%\clearpage
\typeout{END_CHAPTER "index-" \theabspage}
%\newpage


\end{document}